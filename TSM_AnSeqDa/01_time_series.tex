
\section{Basics}

\textbf{Stochastic process (SP)}: sequence of random variables indexed in time order: $ Y_1, Y_2, \ldots, Y_t, \ldots $

Realization of an SP yields a time series: $ \hat{Y}_1=y_1, \hat{Y}_2=y_2, \ldots, \hat{Y}_t=y_t, \ldots $

\begin{minipage}[t]{0.55\textwidth}
\textbf{Statistical forecasting notation:}\\
\begin{tabular}{ll}
    Past observations & $I = \{\hat{Y}_1=y_1, \hat{Y}_2=y_2, \ldots, \hat{Y}_{t-1}=y_{t-1}\}$ \\
    Conditioning & $Y_t | I$ \\
    Point forecast & \(\mean(Y_t | I)\) \\
    Forecast variance & \(\var(Y_t | I)\) \\
    Conditioning on time series & $Y_{t|t-1} = Y_t | \{y_1, y_2, \ldots, y_{t-1}\}$ \\
    \(h\)-step forecast & \(Y_{T+h|t-1} = \E[Y_{T+h} | \{y_1, y_2, \ldots, y_{t-1}\}]\) \\
\end{tabular}
\end{minipage}
\hfill
\begin{minipage}[t]{0.40\textwidth}
\textbf{Time series expressions:}\\
\begin{tabular}{ll}
    Trend & long-term increase/decrease \\
    Seasonal & predictable, seasonal changes \\
    Cyclic & changes that are not of a fixed period \\
\end{tabular}
\end{minipage}

\begin{minipage}[t]{0.5\textwidth}
\textbf{Autocorrelation function} (aka Autocorrelogram):\\
\begin{tabular}{ll}
    Seasonality & High positive values at multiples of seasonality \\
    Trend & Higher absolute values for small lags \\
    White Noise & All values below critical threshold of \(\pm 1.96 / \sqrt{T}\) \\
\end{tabular}
\end{minipage}
\hfill
\begin{minipage}[t]{0.46\textwidth}
\textbf{Statistical Hypothesis Testing (SHT)}:
\begin{itemize}
    \item Test time series against null hypothesis (e.g.\ white noise)
    \item Statistical test (e.g.\ Ljung-Box) yields p-Value
    \item Very small p-Value \(\Leftrightarrow\) reject null hypothesis
\end{itemize}
\end{minipage}


\section{Forecasting}
forecast $ \hat{y}_{t|t-1} $ $\Leftrightarrow$ fitted value of $ y $ at time $ t $ given past values up to time $ t - 1 $
We can perform $h$ step forecasts by iteratively applying one step forecasts.
There are different models to perform forecasts.

\subsection{Simple Forecast Methods}
Average: Forecast of all future values is equal to mean of historical data.

Naïve: Forecast of all future values is equal to the last observed value.

Seasonal Naïve : Forecast of all future values is equal to the last observed value from the same season.

Drift: Forecast equal to last value plus average change in historical values. Results in a straight line between last observed and last forecasted values.

\subsection{Residual Analysis}
Residuals in forecasting are defined as the difference between observed value and its fitted value: $ e_t = y_t - \hat{y}_{t|t-1} $.
Residuals should have zero mean. If they do not, the forecast performed with the fitted model is biased. The should also be uncorrelated.
If they are not, the residuals contain information not captured by the fitted model. Residuals are also assumed to have constant variance and to be normally distributed.

\subsection{Evaluating Forecast Accuracy}
The residuals produced by fitting a model should be indistinguishable from white noise by fulfilling all criteria above.
We can use SHT with a test like Ljung Box to test whether this is true.
Small p-values lead to rejecting the null hypothesis meaning residuals are correlated.
Having small residuals does not guarantee good forecasting performance.

Forecast Errors are also used for evaluating forecasts. The data is split into a training and a test set.
The training set is used for fitting the chosen model while the test set is used to evaluate its performance.
Based on the test set containing $ n $ values, different error metrics can be computed:
$ MAE = \frac{1}{n} \sum_{i=1}^{n} |y_i - \hat{y}_i| $,
$ MSE = \frac{1}{n} \sum_{i=1}^{n} (y_i - \hat{y}_i)^2 $,
$ MAPE = \frac{1}{n} \sum_{i=1}^{n} \left| \frac{y_i - \hat{y}_i}{y_i} \right| \times 100 $,
$ RMSE = \sqrt{\frac{1}{n} \sum_{i=1}^{n} (y_i - \hat{y}_i)^2} $,
$ MASE = \frac{1}{n} \sum_{i=1}^{n} \frac{|y_i - \hat{y}_i|}{\frac{1}{n-1} \sum_{i=2}^{n} |y_i - y_{i-1}|} $,
$ SSE = \sum_{t=1}^{n} (y_t - \hat{y}_t)^2 $.
$ MAE $, $ MSE $, $ RMSE $ and $ SSE $ are all scale dependent.
$ MAPE $ is scale independent but is only sensible if $ y_t \gg 0 \text{ for all } t $, and $ y $ has a natural zero.

Cross-validation for time series involves iteratively splitting a time-ordered dataset into training and test sets,
ensuring that the test set comes after the corresponding training set,
to assess the performance of a model on multiple folds and mitigate the risk of temporal information leakage
(giving the model access to the future it is supposed to predict).

\subsection{Prediction Intervals}
A prediction interval gives a region within which we expect the forecast value with a specific probability.
It serves as a measure of confidence in the prediction. The 95\% prediction interval for an $ h $-step forecast can be computed
as $ \hat{y}_{T+h|T} \pm 1.96 \hat{\sigma}_h $, with $ \hat{\sigma}_h $ as the stddev of the $ h $-step distribution.
Assuming the residuals are normally distributed, $ \hat{\sigma}_h $ can be estimated for different forecast methods as:

$ \text{Mean: } \hat{\sigma} \cdot \sqrt{1+1/T}, \text{Naïve: } \hat{\sigma} \cdot \sqrt{h},
\text{Seasonal Naïve: } \hat{\sigma} \cdot \sqrt{k+1}, \text{Drift: } \hat{\sigma} \cdot \sqrt{h \cdot (1+h/T)} $
where $ \hat{\sigma} $ is the stddev of the residuals, $ T $ is the number of historical values, $ h $ is the integer part of $ (h-1)/m $ and $ m $ is the seasonal period.

\section{Exponential Smoothing}
Exponential smoothing is a time series forecasting method that assigns exponentially decreasing weights to past observations,
providing a weighted average of historical data to generate a smoothed forecast with a focus on recent trends.
Parameters such as $ \alpha $ or $ \phi $ can be found via optimization of the $ SSE $.

\subsection{Simple Exponential Smoothing (SES)}
A a weighted moving average, whose weights decrease exponentially:
$ \hat{y}_{t} = \alpha \cdot y_{t-1} + \alpha \cdot (1 - \alpha) \cdot \hat{y}_{t-1} + \alpha \cdot (1 - \alpha)^2 \cdot \hat{y}_{t-2} + ... $
where $ 0 \le \alpha \le 1 $. The weights sum up to one as a geometric series.
SES provides flat forecasts for all time horizons as it does not incorporate any trend or cyclic components.

Forecast Equation: $ \hat{y}_{t+h|t} = \ell_t $;
Smoothing Equation: $ \ell_t = \alpha \cdot y_t + (1 - \alpha) \cdot \ell_{t-1} $

\subsection{Trend Methods}
\textbf{Holt's linear trend}
Forecast Equation: $ \hat{y}_{t+h|t} = \ell_t + h \cdot b_{t-1} $;
Smoothing Equation: $ \ell_t = \alpha \cdot y_t + (1 - \alpha) \cdot (\ell_{t-1} + b_{t-1}) $;
Trend Equation: $ b_t = \beta \cdot (\ell_t - \ell_{t-1}) + (1 - \beta) \cdot b_{t-1} $

\textbf{Holt's linear with damped trend}
Forecast Equation: $ \hat{y}_{t+h|t} = \ell_t + h \cdot \phi \cdot b_{t-1} $;
Smoothing Equation: $ \ell_t = \alpha \cdot y_t + (1 - \alpha) \cdot (\ell_{t-1} + \phi \cdot b_{t-1}) $;
Trend Equation: $ b_t = \beta \cdot (\ell_t - \ell_{t-1}) + (1 - \beta) \cdot \phi \cdot b_{t-1} $

\subsection{Seasonal Methods}
Holt-Winters is an extension to Holt's method to capture seasonality.

\textbf{Additive:} seasonality component is constant

Forecast Equation: $ \hat{y}_{t+h|t} = \ell_t + hb_{t-1} + s_{t-m+h-m(q+1)} $;
Smoothing Equation: $ \ell_t = \alpha (y_t - s_{t-m}) + (1 - \alpha)(\ell_{t-1} + b_{t-1}) $;
Trend Equation: $ b_t = \beta (\ell_t - \ell_{t-1}) + (1 - \beta) b_{t-1} $;
Seasonality Equation: $ s_t = \gamma (y_t - \ell_{t-1} - b_{t-1}) + (1 - \gamma) s_{t-m} $;

\textbf{Multiplicative}: seasonality component is proportional to series level

Forecast Equation: $ \hat{y}_{t+h|t} = (\ell_t + hb_{t-1}) \cdot s_{t-m+h-m(q+1)} $;
Smoothing Equation: $ \ell_t = \alpha \frac{y_t}{s_{t-m}} + (1 - \alpha)(\ell_{t-1} + b_{t-1}) $;
Trend Equation: $ b_t = \beta (\ell_t - \ell_{t-1}) + (1 - \beta) b_{t-1} $;
Seasonality Equation: $ s_t = \gamma \frac{y_t}{\ell_{t-1} + b_{t-1}} + (1 - \gamma) s_{t-m} $;

\subsection{Exponential Smoothing State Space Models (ETS)}
Contrary to pure forecasting methods, State Space Models can provide point forecasts as well as prediction intervals.
They assume three components that make up the time series:

Error: {Additive, Multiplicative}

Trend: {None, Additive, Additive damped}

Sesonal: {None, Additive, Multiplicative}

\noindent
Automatic Forecasting aims to find the best performing model out of all possible permutations above by choosing
the model with the lowest $ AIC = -2 \ln(\mathcal{L}) + 2k $ or $ AICc = AIC + \frac{2k(k+1)}{n-k-1} $,
where $ \mathcal{L} $ is the likelihood of the model, $ k $ is the number of estimated parameters in the model and $ n $ is the number of observations.

\section{Autoregressive Integrated Moving Average (ARIMA)}
Combines autoregressive (AR) and moving average (MA) components along with differencing (I).

\subsection{Random Walk (with drift)}
$ y_t = c + y_{t-1} + \epsilon_t $

\subsection{Stationarity}
Refers to a time series property where statistical properties, such as mean and variance, remain constant over time,
providing a stable and predictable behavior. Stationary processes drop quickly to zero and no seasonality visible in the ACF.
Non-stationary processes decrease more slowly or not at all.

\subsection{Differencing}
Change between observations in the time series. Can be done multiple times or for observations in the same season.
Differencing can help with obtaining a stationary time series. Finding the correct number of differences to obtain stationarity can be automated.
Backshift notation: $ By_t = y_{t-1} \text{; } y_t' = y_t - y_{t-1} = (1 - B) \cdot y_t \text{; } y_t'' = (1 - B)^2 \cdot y_t$

\subsection{Autoregressive (AR) and Moving Average (MA) models}
Autoregressive Models use multiple regression with lagged values of $ y_t $ as predictors:
$ y_t = c + \phi_1 y_{t-1} + \phi_2 y_{t-2} + \ldots + \phi_p y_{t-p} + \varepsilon_t $, denoted as AR(p)

They work only on stationary data. All $ \phi $ values are restricted such that the complex roots of the characteristic polynom lie outside the unit circle.

Moving Average Models use multiple regression with past noise as predictors:
$ y_t = c + \varepsilon_t - \theta_1 \varepsilon_{t-1} - \theta_2 \varepsilon_{t-2} - \ldots - \theta_q \varepsilon_{t-q} $, denoted as MA(p)

Any $ MA(q) $ process can be written as an $ AR(\infty) $ when the roots of the characteristic polynom lie outside the unit circle.

\subsection{ARMA and ARIMA models}
ARMA: $ y_t = c + \phi_1 y_{t-1} + \phi_2 y_{t-2} + \ldots + \phi_p y_{t-p} + \theta_1 \varepsilon_{t-1} + \theta_2 \varepsilon_{t-2} + \ldots - \theta_q \varepsilon_{t-q} $

ARIMA: Combine ARMA with differencing; p = order of the autoregressive part; d = degree of first differencing involved; q = order of the moving average part
