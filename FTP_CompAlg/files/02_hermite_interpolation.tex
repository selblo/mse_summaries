\section{Hermite Interpolation (Osculation)}
Extension of divided differences: Now, derivatives at data points are also possible as conditions.

From the definition of $y(x)$, we have $y(x_0, x_0) = y'(x_0)$. Generalized, this yields

\begin{minipage}{9cm}

\[y(\underbrace{x_0, \ldots x_n, x_{n+1}}_{(n+2)}) = \frac{y^{(n+1)}(\xi)}{(n+1)!}\]
\[y(\underbrace{x_0, \ldots x_0}_{(n+1)}) = \lim_{\xi \rightarrow x_0} \frac{y^{(n)}(\xi)}{(n)!} = \frac{y^{(n)}(x_0)}{n!}\]
\end{minipage}
\begin{minipage}{6cm}
Bsp. HS13/14 \\
\[ y(x_0,x_1,x_1) = \frac{y(x_1,x_1) - y(x_1,x_0)}{x_1 - x_0} = \frac{y'(x_1) - \frac{y(x_1) - y(x_0)}{x_1 - x_0}}{x_1 - x_0}\]
\end{minipage}


For Hermite interpolation, the same Newton tables are used, but with repetitions (see the example).
The values marked in red are calculated as usual.

It is important that no gaps occur in the calculation of the divided differences
(derivatives must be continuously available, e.g., $y', y'', y'''$ good; $y', y'''$ bad)!
Otherwise, the system of equations is unsolvable.
In case of indeterminate results, a variable can be introduced, which can be determined at the end.


\renewcommand{\arraystretch}{1.0}
\begin{minipage}{10cm}
	\begin{tabular}{|c|lll|}
		\hline
		$x$		&\multicolumn{3}{l|}{$y$}\\
		\hline
		$x_0=2$	&$y(x_0)=1$	&							&\\
				&			&$\frac{y^{(1)}(x_0)}{1!}=1$&\\
		$x_0=2$	&$y(x_0)=1$	&							&$\frac{y^{(2)}(x_0)}{2!}=0$\\
				&			&$\frac{y^{(1)}(x_0)}{1!}=1$&\\
		$x_0=2$	&$y(x_0)=1$	&							&\\
				&			&							&\\
		$x_1=4$	&$y(x_1)=2$	&							&\\
				&			&$\frac{y^{(1)}(x_1)}{1!}=0$&\\
		$x_1=4$	&$y(x_1)=2$	&							&$\frac{y^{(2)}(x_1)}{2!}=0$\\
				&			&$\frac{y^{(1)}(x_1)}{1!}=0$&\\
		$x_1=4$	&$y(x_1)=2$	&							&\\
		\hline
	\end{tabular}
\end{minipage}
\hfill
\newcommand{\mycbox}[1]{\textcolor{red}{#1}}
\begin{minipage}{10cm}
	\begin{tabular}{|c|llllll|}
		\hline
		$x$	&\multicolumn{6}{l|}{$y$}\\
		\hline
		$2$	&\kreisS{$1$}{$a_0$}&			&			&			&				&\\
			&		&\kreisS{$1$}{$a_1$}		&			&			&				&\\
		$2$	&$1$	&			&\kreisS{$0$}{$a_2$}		&			&				&\\
			&		&$1$		&			&\kreisM{\mycbox{$-\frac 18$}}{$a_3$}&				&\\
		$2$	&$1$	&			&\mycbox{$-\frac 14$}&			&\kreisM{\mycbox{$\frac 1{16}$}}{$a_4$}	&\\
			&		&\mycbox{$\frac 12$}	&			&\mycbox{$0$}		&				&\kreisS{\mycbox{$0$}}{$a_5$}\\
		$4$	&$2$	&			&\mycbox{$-\frac 14$}&			&\mycbox{$\frac 1{16}$}	&\\
			&		&$0$		&			&\mycbox{$\frac 18$}	&				&\\
		$4$	&$2$	&			&$0$		&			&				&\\
			&		&$0$		&			&			&				&\\
		$4$	&$2$	&			&			&			&				&\\
		\hline
	\end{tabular}
\end{minipage}\\
\renewcommand{\arraystretch}{1.5}

\newpage

\textbf{Modified} Newton polynomials are used. In this case:

\begin{center}
    \begin{tabular}{ll}
    \toprule
        $\pi_0 = 1$ & $\pi_4 = (x-x_0) (x-x_0) (x-x_0) (x-x_1)$ \\
        $\pi_1 = (x-x_0)$ & $\pi_5 = (x-x_0) (x-x_0) (x-x_0) (x-x_1) (x-x_1)$ \\
        $\pi_2 = (x-x_0) (x-x_0)$ & $\pi_6 = (x-x_0) (x-x_0) (x-x_0) (x-x_1) (x-x_1) (x-x_1)$ \\
        $\pi_3 = (x-x_0) (x-x_0) (x-x_0)$ & \\
    \bottomrule
    \end{tabular}
\end{center}

\begin{align}
p_2(x)	&=a_0\cdot \pi_0+a_1\cdot \pi_1+a_2\cdot \pi_2+a_3\cdot \pi_3+a_4\cdot \pi_4+a_5\cdot \pi_5\nonumber\\[0.3cm]
		&=a_0\cdot 1+a_1\cdot (x-x_0)+a_2\cdot (x-x_0)^2+a_3\cdot (x-x_0)^3+a_4\cdot (x-x_0)^3(x-x_1)+a_5\cdot (x-x_0)^3(x-x_1)^2\nonumber\\[0.3cm]
		&=1+(x-2)-\frac 18(x-2)^3+\frac 1{16} (x-2)^3(x-4)\nonumber
\end{align}

\stepcounter{subsection}\stepcounter{subsection}
\subsection{Error Formula}

\[y(x) - p(x) = \frac{y^{(d)}(\xi)}{d!}(x-x_0)^{d_0}(x-x_1)^{d_1}\cdots (x-x_n)^{d_n}\]
\[x, \xi \in \{\text{min } x_i, \text{max } x_i\}, \quad i = 0, 1, \ldots, 2\]

\(d\) is the total number of conditions, and \(d_i\) is the number of conditions per support point \(x_i\). (\(d = \sum_{i=0}^n d_i\))

\subsection{Missing Derivatives}

In case of missing derivative specifications, variables are introduced, and the tableau is computed with these. In the end, the missing variables are determined from the back.

\textbf{Example:}

Desired: 2nd order polynomial passing through points \(y(2)=1\) and \(y(4)=1\) with the derivative \(y'(4)=1\).

\begin{minipage}[c]{6cm}
	\renewcommand{\arraystretch}{1.0}
	\begin{tabular}{|c|llll|}
		\hline
		\(x\) & \multicolumn{4}{l|}{\(y\)} \\
		\hline
		2 & \kreisS{1}{\(a_0\)} & & & \\
		& & \kreisS{\(\beta\)}{\(a_1\)} & & \\
		2 & 1 & & \kreisB{\(-\frac \beta2\)}{\(a_2\)} & \\
		& & 0 & & \kreisB{\(\frac{1+\beta}{4}\)}{\(a_3\)}\(\overset{!}{=}0\) \\
		4 & 1 & & \(\frac 12\) & \\
		& & 1 & & \\
		4 & 1 & & & \\
		\hline
	\end{tabular}
	\renewcommand{\arraystretch}{1.5}
\end{minipage}
\hfill
\begin{minipage}[c]{12cm}

	\vspace{0.5cm}

	\begin{itemize}
		\item Because the polynomial must have order \(d=2\), we have \(\frac{1+\beta}4=0\).\\
		From this, we get \(\beta=-1\).
		\item The solution polynomial is:
		\begin{align*}
			p(x) &= a_0 + a_1(x-x_0) + a_2(x-x_0)^2 + a_3(x-x_0)^2(x-x_1) \\
			&= 1 + \beta(x-x_0) - \frac \beta2(x-x_0)^2 + \frac{1+\beta}4(x-x_0)^2(x-x_1)\big|_{\beta=-1} \\
			&= 1 - (x-2) + \frac 12(x-2)^2
		\end{align*}
	\end{itemize}
\end{minipage}

\vfill
