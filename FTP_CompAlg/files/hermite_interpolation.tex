\section{Hermite Interpolation (Osculation)}
Erweiterung der dividierten Differenzen: Es sind jetzt auch Ableitungen an Messpunkten als
Bedingungen möglich.

Aus der Definition von $y(x)$ ist $y(x_0,x_0) = y'(x_0)$. Verallgemeinert ergibt das

\begin{minipage}{9cm}

\[y(\underbrace{x_0, \ldots x_n, x_{n+1}}_{(n+2)}) = \frac{y^{(n+1)}(\xi)}{(n+1)!}\]
\[y(\underbrace{x_0, \ldots x_0}_{(n+1)}) = \lim_{\xi \rightarrow x_0} \frac{y^{(n)}(\xi)}{(n)!} = \frac{y^{(n)}(x_0)}{n!}\]
\end{minipage}
\begin{minipage}{6cm}
Bsp. HS13/14 \\
\[ y(x_0,x_1,x_1) = \frac{y(x_1,x_1) - y(x_1,x_0)}{x_1 - x_0} = \frac{y'(x_1) - \frac{y(x_1) - y(x_0)}{x_1 - x_0}}{x_1 - x_0}\]
\end{minipage}


Für die Hermite Interpolation werden dieselben Newton Tableaus verwendet, aber mit Wiederholungen (siehe Beispiel).
Die rot markierten Werte werden wie üblich berechnet.

Wichtig ist, dass bei der Berechnung der dividierten Differenzen keine Löcher entstehen (Ableitungen müssen Lückenlos vorhanden sein (z.B.: $y',y'',y'''$ good; $y',y'''$ bad))! Dann ist das
Gleichungssystem nicht lösbar! Es kann bei unbestimmten Resultaten eine Variable eingesetzt werden,
welche am Schluss ermittelt werden kann.

\renewcommand{\arraystretch}{1.0}
\begin{minipage}{10cm}
	\begin{tabular}{|c|lll|}
		\hline
		$x$		&\multicolumn{3}{l|}{$y$}\\
		\hline
		$x_0=2$	&$y(x_0)=1$	&							&\\
				&			&$\frac{y^{(1)}(x_0)}{1!}=1$&\\
		$x_0=2$	&$y(x_0)=1$	&							&$\frac{y^{(2)}(x_0)}{2!}=0$\\
				&			&$\frac{y^{(1)}(x_0)}{1!}=1$&\\
		$x_0=2$	&$y(x_0)=1$	&							&\\
				&			&							&\\
		$x_1=4$	&$y(x_1)=2$	&							&\\
				&			&$\frac{y^{(1)}(x_1)}{1!}=0$&\\
		$x_1=4$	&$y(x_1)=2$	&							&$\frac{y^{(2)}(x_1)}{2!}=0$\\
				&			&$\frac{y^{(1)}(x_1)}{1!}=0$&\\
		$x_1=4$	&$y(x_1)=2$	&							&\\
		\hline
	\end{tabular}
\end{minipage}
\hfill
\newcommand{\mycbox}[1]{\textcolor{red}{#1}}
\begin{minipage}{10cm}
	\begin{tabular}{|c|llllll|}
		\hline
		$x$	&\multicolumn{6}{l|}{$y$}\\
		\hline
		$2$	&\kreisS{$1$}{$a_0$}&			&			&			&				&\\
			&		&\kreisS{$1$}{$a_1$}		&			&			&				&\\
		$2$	&$1$	&			&\kreisS{$0$}{$a_2$}		&			&				&\\
			&		&$1$		&			&\kreisM{\mycbox{$-\frac 18$}}{$a_3$}&				&\\
		$2$	&$1$	&			&\mycbox{$-\frac 14$}&			&\kreisM{\mycbox{$\frac 1{16}$}}{$a_4$}	&\\
			&		&\mycbox{$\frac 12$}	&			&\mycbox{$0$}		&				&\kreisS{\mycbox{$0$}}{$a_5$}\\
		$4$	&$2$	&			&\mycbox{$-\frac 14$}&			&\mycbox{$\frac 1{16}$}	&\\
			&		&$0$		&			&\mycbox{$\frac 18$}	&				&\\
		$4$	&$2$	&			&$0$		&			&				&\\
			&		&$0$		&			&			&				&\\
		$4$	&$2$	&			&			&			&				&\\
		\hline
	\end{tabular}
\end{minipage}\\
\renewcommand{\arraystretch}{1.5}

%TODO: achtung hier ist ein newpage
\newpage

Es werden die \textbf{modifizierten} Newton Polynome verwendet. In diesem Fall:

\begin{center}
    \begin{tabular}{ll}
    \toprule
        $\pi_0 = 1$ & $\pi_4 = (x-x_0) (x-x_0) (x-x_0) (x-x_1)$ \\
        $\pi_1 = (x-x_0)$ & $\pi_5 = (x-x_0) (x-x_0) (x-x_0) (x-x_1) (x-x_1)$ \\
        $\pi_2 = (x-x_0) (x-x_0)$ & $\pi_6 = (x-x_0) (x-x_0) (x-x_0) (x-x_1) (x-x_1) (x-x_1)$ \\
        $\pi_3 = (x-x_0) (x-x_0) (x-x_0)$ & \\
    \bottomrule
    \end{tabular}
\end{center}

\begin{align}
p_2(x)	&=a_0\cdot \pi_0+a_1\cdot \pi_1+a_2\cdot \pi_2+a_3\cdot \pi_3+a_4\cdot \pi_4+a_5\cdot \pi_5\nonumber\\[0.3cm]
		&=a_0\cdot 1+a_1\cdot (x-x_0)+a_2\cdot (x-x_0)^2+a_3\cdot (x-x_0)^3+a_4\cdot (x-x_0)^3(x-x_1)+a_5\cdot (x-x_0)^3(x-x_1)^2\nonumber\\[0.3cm]
		&=1+(x-2)-\frac 18(x-2)^3+\frac 1{16} (x-2)^3(x-4)\nonumber
\end{align}

\stepcounter{subsection}\stepcounter{subsection}
\subsection{Fehlerformel}

$$y(x)-p(x)=\frac{y^{(d)}(\xi)}{d!}(x-x_0)^{d_0}(x-x_1)^{d_1}\cdots (x-x_n)^{d_n}\qquad x,\xi \in \{\text{min } x_i,\text{max } x_i\}\quad\text{für}\quad i=0,1,\ldots,2$$
$d$ ist die total Anzahl Bedingungen und $d_i$ die Anzahl Bedingungen pro Stützstelle $x_i$. ($d = \sum_{i=0}^n d_i$)

\subsection{Fehlende Ableitungen}

Bei fehlenden Ableitungsvorgaben werden Variablen eingesetzt und das Tableau mit diesen durchgerechnet. Am Schluss wird von hinten her auf die fehlenden Variablen geschlossen.\\

\textbf{Beispiel:}\\
Gesucht: Polynom 2. Ordnung, welches durch die Punkte $y(2)=1$ und $y(4)=1$ geht und die Ableitung $y'(4)=1$ aufweist.

\begin{minipage}[c]{6cm}
	\renewcommand{\arraystretch}{1.0}
	\begin{tabular}{|c|llll|}
		\hline
		$x$	&\multicolumn{4}{l|}{$y$}\\
		\hline
		$2$	&\kreisS{$1$}{$a_0$}&	&&\\
			&	&\kreisS{$\beta$}{$a_1$}&&\\
		$2$	&$1$&	&\kreisB{$-\frac \beta2$}{$a_2$}&\\
			&	&$0$&				  &\kreisB{$\frac{1+\beta}{4}$}{$a_3$}$\overset{!}{=}0$\\
		$4$	&$1$&	&$\frac 12$&\\
			&	&$1$&&\\
		$4$	&$1$&	&&\\
		\hline
	\end{tabular}
	\renewcommand{\arraystretch}{1.5}
\end{minipage}
\hfill
\begin{minipage}[c]{12cm}

	\vspace{0.5cm}

	\begin{itemize}
		\item Weil das Polynom die Ordnung $d=2$ aufweisen muss, gilt: $\frac{1+\beta}4=0$\\
		Daraus folgt: $\beta=-1$
		\item Das Lösungspolynom ist:
		\begin{align} p(x)&=a_0+a_1(x-x_0)+a_2(x-x_0)^2+a_3(x-x_0)^2(x-x_1)\nonumber\\
		&=1+\beta(x-x_0)-\frac \beta2(x-x_0)^2+\frac{1+\beta}4(x-x_0)^2(x-x_1)\big|_{\beta=-1}\nonumber\\
		&=1-(x-2)+\frac 12(x-2)^2\nonumber
		\end{align}
	\end{itemize}
\end{minipage}

\vfill
