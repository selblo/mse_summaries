
\section{Convolutional Neural Networks (CNNs)}


\subsection{Network Layer Properties}

Given an input shape of \mintinline{python}{(n, n, c)} and ignoring the batch dimension
(\mintinline{python}{keras} default shape is \mintinline[breaklines]{python}{(batch_size, height, width, channels)}).
Convolution layer arguments are \mintinline{python}{filters=f} and \mintinline{python}{kernel_size=k}.

\begin{table}[h!]
    \centering
    \begin{tabular}{@{}lll@{}}\toprule
        Layer & Output shape & \# of params\\ \midrule
        \mintinline{python}{Conv2D(f, k, padding='same')} &
            \mintinline{python}{(n, n, f)} & \(k \cdot k \cdot c \cdot f + f\) \\
        \mintinline{python}{Conv2D(f, k, padding='valid')} &
            \mintinline{python}{(n-k+1, n-k+1, f)} & \(k \cdot k \cdot c \cdot f + f\) \\
        \mintinline{python}{Conv2D(f, k, strides=(sr, sc), padding='same')} &
            \mintinline{python}{(n/sr, n/sc, f)} & \(k \cdot k \cdot c \cdot f + f\) \\
        \mintinline{python}{Conv2DTranspose(f, k, strides=(sr, sc), padding='same')} &
            \mintinline{python}{(sr*n, sc*n, f)} & \(k \cdot k \cdot c \cdot f + f\) \\
        \mintinline{python}{MaxPool2D(pool_size=(sr, sc))} &
            \mintinline{python}{(n/sr, n/sc, c)} & \(0\)\\
        \mintinline{python}{UpSampling2D(size=(sr, sc))} &
            \mintinline{python}{(sr*n, sc*n, c)} & \(0\)\\
        \bottomrule
    \end{tabular}
\end{table}

The required \textbf{memory} of a layer is equal to the product of the output shape of a layer (times whatever the data type requires).

\vspace{3mm}

The \textbf{receptive field} of a given neuron can be calculated as follows:
\begin{enumerate}
    \item Start with the dimension of the last convolutional kernel (e.g.\ \(3 \times 3\)).
    \item For every layer further upp, extend the area as follows:
    \begin{itemize}
        \item For every pooling layer of size \(p \times p\), increase the area dimensions by \(2 \cdot (p-1)\).
        \item For every convolutional layer of kernel size \(k \times k\), increase the area dimensions by \(k-1\).
    \end{itemize}
\end{enumerate}
