$\sigma$ = Sprungfunktion. Wenn $1$ transformiert wird, soll $\sigma$ genommen werden (also im Frequenzbereich $\frac{1}{s}$).
\let\DS=\displaystyle
\renewcommand{\arraystretch}{2.5}
{ \[
\begin{array}{|@{\hspace{1cm}}c@{\hspace{1cm}}c@{\hspace{1cm}}c@{\hspace{1cm}}||c@{\hspace{1cm}}c@{\hspace{1cm}}c@{\hspace{1cm}}c@{\hspace{1cm}}|}
\hline
\sigma(t) & \FT & \DS\frac{1}{s} &&\sigma(t)\cdot t^2\cdot e^{\,\alpha\,t} & \FT & \DS\frac{2}{(s-\alpha)^3} \\
\hline
\sigma(t)\cdot t & \FT & \DS\frac{1}{s^2} &&\sigma(t)\cdot t^n\cdot e^{\,\alpha\,t} & \FT & \DS\frac{n!}{(s-\alpha)^{n+1}} \\
\hline
\sigma(t)\cdot t^2 & \FT & \DS\frac{2}{s^3} &&\sigma(t)\cdot\sin\,(\omega\,t) & \FT & \DS\frac{\omega}{s^2+\omega^2} \\
\hline
\sigma(t)\cdot t^n & \FT & \DS\frac{n!}{s^{n+1}} &&\sigma(t)\cdot\cos\,(\omega\,t) & \FT & \DS\frac{s}{s^2+\omega^2} \\
\hline
\sigma(t)\cdot e^{\,\alpha\,t} & \FT & \DS\frac{1}{s-\alpha} &&\delta(t) & \FT & 1(s) \\
\hline
\sigma(t)\cdot t\cdot e^{\,\alpha\,t} & \FT & \DS\frac{1}{(s-\alpha)^2} &&\delta(t-a) & \FT & e^{-a\,s} \\
\hline
\end{array} \] }
