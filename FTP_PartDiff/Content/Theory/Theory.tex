\section{Theory}

\subsection{Begriffe und Klassifikation}

\subsubsection{Ordnung}

Wie bei gewöhnlichen Differentialgleichungen ist die Ordnung
die höchste Ableitung der unbekannten Funktion, die in der
Differentialgleichung vorkommt.\\

\textbf{PDGL 1. Ordnung: } \qquad$F\biggl(x_1,\dots,x_n, u, \frac{\partial u}{\partial x_1},\dots,\frac{\partial u}{\partial x_n}\biggr)$

Sie kann durch Substitution $\frac{\partial u}{\partial x_i}\to p_i$ durch  $F(x_1,\dots,x_n,u,p_1,\dots,p_n),$
ausgedrückt werden.\\

\textbf{PDGL 2. Ordnung: } \qquad $F\biggl(x_1,\dots,x_n,u,
\frac{\partial u}{\partial x_1},\dots,\frac{\partial u}{\partial x_n},
\frac{\partial^2 u}{\partial x_1^2},\dots,\frac{\partial^2 u}{\partial x_n^2}\biggr)$

Sie kann durch Substitution $\frac{\partial u}{\partial x_i}\to p_i,~\frac{\partial^2 u}{\partial x_i\partial x_j}\to t_{ij}$ durch
$F(x_1,\dots,x_n,u,p_1,\dots,p_n,t_{11},t_{12},\dots,t_{n,n-1},t_{nn})$
ausgedrückt werden.\\
\textbf{Beispiel Übung}\\
\begin{tabular}{lclcl}
$\frac{\partial^2 u}{\partial x_1 \partial x_2} = 0$ & $\Rightarrow$ & 
    $F(t_{12}) = t_{12} = 0$ & $\Leftrightarrow$ &
    $F \left( \frac {\partial^2 u}{\partial x_1 \partial x_2} \right) = 0$ \\
$\frac{\partial u}{\partial x_1} = \frac{\partial u}{\partial x_2} $ & $\Rightarrow$ & 
    $F(p_1,p_2) = p_1 - p_2 = 0$ & $\Leftrightarrow$ & 
    $F \left( \frac{\partial u}{\partial x_1}, \frac{\partial u}{\partial x_2} \right) = 0$ \\
$x_1 \frac{\partial u}{\partial x_1} + x_2 \frac{\partial u}{\partial x_2} = \frac{\partial u}{\partial x_3} $ & 
    $\Rightarrow$ & $F(x_1,x_2,p_1,p_2,p_3) = x_1 p_1 + x_2 p_2 - p_3$ & $\Leftrightarrow$ &
    $F\left( x_1,x_2,\frac{\partial u}{\partial x_1},\frac{\partial u}{\partial x_2},\frac{\partial u}{\partial x_3} \right)
    = x_1\frac{\partial u}{\partial x_1} + x_2\frac{\partial u}{\partial x_2} - \frac{\partial u}{\partial x_3} = 0$

\end{tabular}

\subsubsection{Laplace-Operator}
\begin{tabular}{ll}
Kartesisch: $\Delta u(x,y,z)=\frac{\partial^2u}{\partial x^2}+\frac{\partial^2u}{\partial y^2}+\frac{\partial^2u}{\partial z^2}$
& Zylinder: $\Delta f ( \rho , \phi , z ) = \frac{1}{\rho} \frac{\partial}{\partial \rho}
\left( \rho\,\frac{\partial f}{\partial \rho} \right) +
\frac{1}{\rho^2}\frac{\partial^2 f}{\partial \phi^2} +
\frac{\partial^2 f}{\partial z^2}$ \\
Polar: $\Delta f(r, \varphi ) =
\frac{1}{r}\frac{\partial f}{\partial r} r \frac{\partial f}{\partial r} + \frac{1}{r^2} \frac{\partial^2 f}{\partial \varphi^2}$
& Kugel: $\Delta f ( r , \vartheta , \phi ) = \frac{1}{\rho^2}\frac{\partial}{\partial \rho} \left(\rho^2 \frac{\partial f}{\partial \rho}\right) + \frac{1}{\rho^2 \sin\theta} \frac{\partial}{\partial \theta} \left(\sin\theta \frac{\partial f}{\partial \theta}\right) + \frac{1}{\rho^2 \sin^2\theta} \frac{\partial^2 f}{\partial \varphi^2}$
\end{tabular}

\subsubsection{Umwandlung in System niedriger Ordnung}

\begin{tabular}{ll}
Gegeben:& $F\biggl(x,y,u,\frac{\partial u}{\partial x},\frac{\partial u}{\partial y},
\frac{\partial^2 u}{\partial x^2},\frac{\partial^2 u}{\partial x\partial y},
\frac{\partial^2u}{\partial y^2}\biggr)=0.$\\[0.2cm]
Substitution: & $p=\frac{\partial u}{\partial x},\qquad q=\frac{\partial u}{\partial y}$\\[0.2cm]
Für zweite Ableitungen: & $\frac{\partial^2 u}{\partial x^2}=\frac{\partial p}{\partial x},\quad \frac{\partial^2 u}{\partial x\partial y}=\frac{\partial p}{\partial y}=\frac{\partial q}{\partial x},\quad\frac{\partial^2 u}{\partial y^2}=\frac{\partial q}{\partial y}$\\[0.2cm]
Gleichungssystem 1.Ordnung& $p=\frac{\partial u}{\partial x},\quad q=\frac{\partial u}{\partial y},\quad\frac{\partial p}{\partial y}=\frac{\partial q}{\partial x}$
\end{tabular}

\subsubsection{Notationen einer PDGL, Gebiet $\Omega$}
\begin{minipage}{4cm}
	\begin{tabular}{ll}
	\includegraphics[width=3cm]{Content/Theory/Gebiet}&
	\end{tabular}
\end{minipage}
\begin{minipage}{4cm}	
	\begin{tabular}{ll}
		$\overset{\circ}{\Omega}$ & Innere Punkte\\
		$\partial\Omega$ & Rand\\
		$\overset{\_}{\Omega}$ & Gebiet $\Omega$ und Rand $\partial\Omega$\\
	\end{tabular}
\end{minipage}

Das Gebiet einer PDGL \textbf{muss} offen sein, nur dann ist die partielle
Ableitung überall definiert. Das Gebiet ist offen, wenn um jeden Punkt im Gebiet
$\Omega$ ein kleiner Ball gezeichnet werden kann, welches sich auch im Gebiet $\Omega$ befindet.\\

\begin{minipage}{4cm}
	Kein Gebiet:\\
	\includegraphics[width=4cm]{Content/Theory/gebiet_1.pdf}\\

\end{minipage}
\begin{minipage}{4cm}
  	Gebiet:\\
  	\includegraphics[width=4cm]{Content/Theory/gebiet_2.pdf}\\
\end{minipage}

\textbf{Lösung einer PDGL:}\\
\begin{tabular}{ll}
Gegeben:& Gebiet $\Omega$, PDGL,Randwerte $\partial\Omega$\\
Lösung:& Funktion $u$: $\overset{\_}{\Omega}\rightarrow \mathbb{R}$, PDGL in $\Omega$ und Randwerte auf $\partial\Omega$\\
\end{tabular} \\
'gut gestellt' wen die Angaben die Lösung eindeutig bestimmen

\subsubsection{Klassifikation einer PDGL}
\begin{tabular}{lll}
Ordnung:& \multicolumn{2}{l}{Höchste vorkommende partielle Ableitung}\\
Typ:& Linear: & Linear in $u, x_1,...,x_n, \frac{\partial u}{\partial x_1},\ldots,\frac{\partial u}{\partial x_n}$\\
& Quasilinear: &  Linear in $\frac{\partial u}{\partial x_1},\ldots,\frac{\partial u}{\partial x_n}$\\
& Nichtlineare: & Alles andere
\end{tabular}

\subsection{Methode Charakteristiken}

\textbf{Wichtig:} Als Anfangsbedingungen dürfen \textbf{keine} Charakteristiken verwendet werden, sonst ist die Charakteristik die Lösung (anstatt Fläche ergibt sich eine Kurve).\\
\textbf{Wichtig:} Die Charakteristik darf den Rand nur einmal durchlaufen.\\
Nützlich für Quasilineare PDGL 1.Ordnung. Wenn Separation möglich ist, sollte diese (einfachere) Methode verwendet werden.\\

Ausgangslage:
\[
    a(x,y,u)\cdot\partFrac{u}{x}+b(x,y,u)\cdot\partFrac{u}{y}-c(x,y,u)=0
\]
Charakteristik:
\[
    \frac{d}{dt} \begin{bmatrix} x(t) \\ y(t) \\ u(t) \end{bmatrix}
    = \begin{bmatrix} a(x,y,u) \\ b(x,y,u) \\ c(x,y,u) \end{bmatrix}
\]


\begin{tabular}{ll}
Gebiet:& $\Omega\{\ldots|x>0, \text{alle }y\}$\qquad Randbedingung: $u(0,y_0)=g(y_0)$\\
Vektorielle Schreibweise:& $\begin{bmatrix}
    a(x,y,u)\\ b(x,y,u)\\ c(x,y,u)
    \end{bmatrix}
\underset{\overrightarrow{n}\text{: Normale auf Fläche}}{\underbrace{\begin{bmatrix}
\partFrac{u}{x} & \partFrac{u}{y} & -1
\end{bmatrix}}}=0 $ \\[1cm]
Tangenten:& $\overrightarrow{t}_x=\begin{bmatrix}1\\0\\ \partFrac{u}{x}\end{bmatrix}\qquad 
			\overrightarrow{t}_y=\begin{bmatrix}0\\1\\ \partFrac{u}{y}\end{bmatrix}\qquad \overrightarrow{n} \bullet \overrightarrow{t_x} = 0 \qquad \overrightarrow{n} \bullet \overrightarrow{t_y} = 0 \qquad \overrightarrow{t_x} \bullet \overrightarrow{t_y} = \overrightarrow{n}$\\[1cm]

Lösungsweg: & Für jeden Anfangspunkt $\begin{bmatrix} 0\\y_0\\g(y_0)\end{bmatrix}$ finde eine Charakteristik, diese nach $x$, $y$ auflösen.
\end{tabular}

\paragraph{Randbedingungen}
Eine Lösungsfunktion $u(x,y)$ muss von Charakteristiken überdeckt werden.
Die Lösung wird nun durch die Randwerte bestimmt.

\begin{minipage}{10cm}
    Für das dargestellte Gebiet $\Omega$ sind verschiedene Fälle möglich:
    \begin{enumerate}
        \item Randwerte am \emph{linken} und \emph{rechten} Rand sind vorgegeben.
        Ein Gebiet in der Mitte ist nicht bestimmt.
        \item Randwerte am \emph{oberen} und \emph{unteren} Rand sind vorgegeben.
        Ein Teil des Gebiets ist überbestimmt.
        \item Randwerte am \emph{linken} und \emph{unteren} Rand sind vorgegeben.
        Funktion ist eindeutig bestimmt (aber nicht unbedingt überall differenzierbar).
    \end{enumerate}
    Die Lösung ist also nicht für alle Randwerte bestimmbar. \newline
    Wenn sich zwei Charakteristiken treffen $\rightarrow$ Singularität
\end{minipage}
\hspace{0.5cm}
\begin{minipage}{8cm}
    \centering
    \includegraphics[width=6cm]{Content/Theory/charakteristiken_randwerte.png}
\end{minipage}

\paragraph{Beispiel:}~\\
\begin{enumerate}
	\item PDGL mit Randbedingungen und Definitionsbereich: $\partFrac ux+2\partFrac uy=3$, \; $u(0,y)=g(y)=\sin(y) \Rightarrow u(0,y_0) = g(y_0) = \sin(y_0)$\\
	Terme in Matrixschreibweise: $\begin{bmatrix}a\\b\\c\end{bmatrix}=\begin{bmatrix}1\\2\\3\end{bmatrix}$
	\item Charakteristiken ausrechnen PDGL $\rightarrow$ DGL: 	$\frac {d}{dt}\begin{bmatrix}x(t)\\y(t)\\u(t)\end{bmatrix}=\begin{bmatrix}1\\2\\3\end{bmatrix}$
	\item DGL's lösen (für Standard-DGL's, siehe \ref{sec:dgls} auf Seite \pageref{sec:dgls}.): 
	$\begin{bmatrix}x\\y\\u\end{bmatrix}=\begin{bmatrix}1t+x_0\\2t+y_0\\3t+u_0\end{bmatrix}$
	\item Anfangsbedingungen einsetzen: $\begin{bmatrix}x\\y\\u\end{bmatrix}=\begin{bmatrix}1t+x_0\\2t+y_0\\3t+u_0\end{bmatrix}\Bigg|_{t=0}=
	\begin{bmatrix}x_0\\y_0\\u_0\end{bmatrix}=\begin{bmatrix}0\\y_0\\\sin(y_0)\end{bmatrix}$\\
	Lösung der DGL ist: $\begin{bmatrix}x\\y\\u\end{bmatrix}=\begin{bmatrix}1\\2\\3\end{bmatrix}\cdot t+ \begin{bmatrix}0\\y_0\\\sin(y_0)\end{bmatrix}$\\
	
	\item Eliminieren aller Variablen ausser $u,x,y$: $u=3x+\sin(y-2x)$
	\item Kontrolle:
	Resultat ($u=3x+\sin(y-2x)$) ableiten und in Aufgabenstellung einsetzen $\partFrac ux+2\partFrac uy=3$ und schauen ob es erfüllt.
	
\end{enumerate}

\subsection{Methode Separation}
Wahl eines geeigneten Koordinatensystems ist wichtig.


\begin{enumerate}
\item \textbf{Ansatz} (Höchste Ableitung ausschlaggebend): 
	\begin{itemize}
		\item Für PDGL 1.Ordnung: $U(x,y)=X(x) + Y(y)$
		\item Für PDGL 2.Ordnung: $U(x,y)=X(x) \cdot Y(y)$ 
	\end{itemize}
\item \textbf{Einsetzen: } Ansatz in PDGL einsetzen.
\item \textbf{Separation: } Auf jeder Seite der PDGL darf nur noch eine Variable vorkommen. Die beiden jetzt gewöhnlichen DGL sind über eine Konstante gekoppelt (fixieren der Variable). Wahl der Konstante: Wenn Schwingung erwartet wird: $-k^2$, sonst $k$, ausser man weiss es besser ;-).
\item \textbf{Lösen der DGL's: } Man erhält eine Familie von Lösungen	
\item \textbf{Gesamtlösung "'Zusammenbasteln"': } (Linearkombination der Lösungen), Randbedingungen einhalten!
\end{enumerate}

\begin{minipage}{0.49\textwidth}
\textbf{Beispiel 1: } PDGL: $\frac1x\partFrac{u}{x}+\frac1y\partFrac{u}{y}=\frac{1}{y^2}$
\begin{enumerate}
	\item Ansatz:\\[0.4cm]
	$u(x,y)=X(x) + Y(y)$ (1.Ordnung)
	\item Einsetzen:\\[0.4cm]
	$\partFrac{u}{x}=X'(x)$\qquad $\partFrac{u}{y}=Y'(y)$ \quad $\Rightarrow$ \quad $\frac{X'(x)}{x}+\frac{Y'(y)}{y}=\frac{1}{y^2}$
	\item Separation:\\[0.4cm]
	$\frac{X'(x)}{x}=k=\frac{1}{y^2}-\frac{Y'(y)}{y}$
	\item DGL'2 lösen:\\[0.4cm]
	$X'(x)=k\cdot x \quad\Rightarrow\quad X(x)=\frac12 kx^2+C_x$\\
	$Y'(y)=\frac1y-ky \quad\Rightarrow\quad Y(y)=\ln(y)-\frac12 ky^2+C_y$
	\item Linearkombination:\\[0.4cm]
	$u(x,y)=\frac12 kx^2 - \frac12 ky^2+ln(y)+C$
\end{enumerate}

\textbf{Beispiel 2: } PDGL: $x^2\partFrac{^2u}{x^2}+x\partFrac{u}{x}+y^2\partFrac{^2u}{y^2}+y\partFrac{u}{y}=0$\\ 
Randbedingungen: $\Omega=[1,2]\times[1,2]$ \qquad $u=0$ auf $\partial\Omega$
\begin{enumerate}
	\item Ansatz:\\[0.4cm]
	$u(x,y)=X(x) \cdot Y(y)$ (2.Ordnung)
	\item Einsetzen:\\[0.4cm]
	$x^2X''(x)Y(y)+xX'(x)Y(y)+y^2X(x)Y''(y)+yX(x)Y'(y)=0$
	\item Separation: Division durch $X(x)Y(y)$\\[0.4cm]
	$\frac{x^2X''(x)}{X(x)}+\frac{xX'(x)}{X(x)}+\frac{y^2Y''(y)}{Y(y)}+\frac{yY'(y)}{Y(y)}=0\quad\Rightarrow\quad \frac{x^2X''(x)}{X(x)}+\frac{xX'(x)}{X(x)}=k=-\frac{y^2Y''(y)}{Y(y)}-\frac{yY'(y)}{Y(y)}$
	\item DGL'2 lösen:\\[0.4cm]
	$\frac{x^2X''(x)}{X(x)}+\frac{xX'(x)}{X(x)}=k\quad\Rightarrow\quad x^2X''(x)+xX'(x)-kX(x)=0$\qquad mit $X(1)=X(2)=0$\\
	$\frac{y^2Y''(y)}{Y(y)}-\frac{yY'(y)}{Y(y)}=-k\quad\Rightarrow\quad y^2Y''(y)+yY'(y)+kY(y)=0$\qquad mit $Y(1)=Y(2)=0$\\[0.4cm]
	Lösung der DGL hier nicht gemacht.
\end{enumerate}
\end{minipage}
\hfill
\begin{minipage}{0.49\textwidth}
\textbf{Beispiel 3: }PDGL: $\partFrac{^2u}{t^2}=\partFrac{^2u}{x^2}$ \quad $u(t=0,x)=0$\\
Randbedingungen: $x=[0,\pi]$ \\
\qquad\qquad $\partFrac{u}{t}(t=0,x)=\sin^3(x)=\frac34\sin(x)-\frac14\sin(3x)$
\begin{enumerate}
	\item Ansatz:\\[0.4cm]
	$u(t,x)=T(t) \cdot X(x) $ (2.Ordnung)
	\item Einsetzen:\\[0.4cm]
	$T''(t)\cdot X(x) = X''(x)\cdot T(t)$
	\item Separation:\\[0.4cm]
	$\frac{X''(x)}{X(x)}= -\mu^2=\frac{T''(t)}{T(t)}$
	\item DGL'2 lösen:\\[0.4cm]
		$X(x)=\sin(\mu x) \qquad T(t)=\sin(\mu t)$\\
		$X(x)=\cos(\mu x) \qquad T(t)=\cos(\mu t)$
	\item Linearkombination:\\[0.4cm]
		Die Randbedingungen $x=0$ und $x=\pi$ können nur mit $\sin(\mu x) $ und  positivem, ganzzahligen $\mu$ erfüllt werden. $cos(nx)$-Therme fallen weg.\\[0.4cm]
		$u(t,x)=\sum\limits_{n=1}^{\infty}{a_n\sin(nx)\sin(nt)} + \sum\limits_{n=1}^{\infty}{b_n\sin(nx)\cos(nt)}$\\[0.4cm]
		Die Koeffizienten $a_n$ und $b_n$ müssen mit Hilfe der Anfangsbedingungen zur Zeit $t=0$ bestimmt werden:\\[0.4cm]
		$u(0,x)=\sum\limits_{n=1}^{\infty}{b_n\sin(nx)}=0 \quad\Rightarrow\quad b_n=0$\\[0.2cm]
		$\partFrac{u}{t}(\pi,x)=\sum\limits_{n=1}^{\infty}{a_nn\sin(nx)}=\sin^3(x)=\frac34\sin(x)-\frac14\sin(3x) \quad\Rightarrow\quad a_1=\frac34 \quad a_3=-\frac{1}{12}\quad a_k=0$ für $k\neq 1,3$\\[0.4cm]
		$u(t,x)=\frac34\sin(x)\sin(t)-\frac1{12}\sin(3x)\sin(3t)$
\end{enumerate}
\end{minipage}





\subsection{Hamilton-Jacobi Theorie}
Die Hamilton-Jacobi Theorie geht von einer Gesamtenergie $H(x_i,p_i)$ in Abhängigkeit von Ort
und Impuls aus.
Dazu muss eine Funktion $S(x_i,t)$ gefunden werden, für welche
\begin{align*}
    \frac{\partial S}{\partial t} = H\left(x_i,p_i\right) = H\left(x_i,\frac{\partial S}{\partial x_i}\right)
    && \text{mit} \quad
    p_i = \frac{\partial S}{\partial x_i}
\end{align*}
Diese kann meist durch Integration gelöst werden.
Dabei werden die Integrationskonstanten $P_i$ eingeführt.
Die \emph{Bahnparameter} $Q_i$ sind
\[
    Q_i = \frac{\partial S}{\partial P_i}
\]
und die Bahnkurve hat die Form
\[
    x_i(t,Q_i,P_i)
\]


\subsection{Transformations}
\begin{itemize}
\item The transition from functions to Fourier series transforms a partial differential equation into a family of ordinary differential equations for the individual Fourier coefficients.
\item Integral transformations can transform a partial differential equation into a family of partial differential equations with fewer variables or even ordinary differential equations.
\item Integral transformations and their inverses can provide formulas for solutions to certain partial differential equations, answering the question of well-posedness for given boundary conditions.
\end{itemize}

\begin{tabular}{ll}
  Domain & Transformation \\
  \hline
  $[0, \infty)$ & Laplace transformation \\
  $\mathbb{R}$ & Fourier transformation \\
  $[- \pi, \pi]$ & Fourier series \\
\end{tabular}

\subsubsection{Fourier Series}
$\boxed{u(t,x)=\frac{a_0(t)}{2}+\sum\limits_{k=1}^{\infty}{a_k(t)\cos(kx)+b_k(t)\sin(kx)}}$\\[0.4cm]

\subsubsection{Example: Vibrating String}

$\boxed{\partial_t^2u=\partial_x^2u}$

\begin{enumerate}
\item Insert Fourier analysis approach into PDE:\\
$$\partial_t^2(t,x)=\frac{a_0''(t)}{2}+\sum\limits_{k=1}^{\infty}{a_k''(t)\cos(kx)+b_k''(t)\sin(kx)}
\qquad \qquad
\partial_x^2(t,x)=-\sum\limits_{k=1}^{\infty}{a_k(t)k^2\cos(kx)+b_k(t)k^2\sin(kx)}$$
$$\partial_t^2(t,x)=\partial_x^2(t,x)
\qquad \Longleftrightarrow \qquad \frac{a_0''(t)}{2}+\sum\limits_{k=1}^{\infty}{a_k''(t)\cos(kx)+b_k''(t)\sin(kx)}=-\sum\limits_{k=1}^{\infty}{a_k(t)k^2\cos(kx)+b_k(t)k^2\sin(kx)}$$
$\boxed{\Rightarrow\quad \frac{a_0''(t)}{2}+\sum\limits_{k=1}^{\infty}{\big(a_k''(t)+a_k(t)k^2\big)\cos(kx)+\big(b_k''(t)+b_k(t)k^2\big)\sin(kx)}=0}$
\item This equation is only solvable if all coefficients vanish (Fourier theory):\\[0.2cm]
$a_0''(t)=0 \qquad a_k''(t)=-k^2a_k(t)\qquad b_k''(t)=-k^2b_k(t)$
\item The Fourier transformation has transformed the PDE into a system of ODEs, the solutions of which are well-known:\\[0.2cm]
$a_0(t)=m_0(t)+c_0\qquad a_k(t)=A_k^a\cos(kt)+B_k^a\sin(kt)\qquad b_k(t)=A_k^b\cos(kt)+B_k^b\sin(kt)$
\end{enumerate}

\textbf{Initial Conditions:}
The differential equations for the coefficients $a_k(t)$ and $b_k(t)$ can only be completely solved when initial or boundary conditions are given.
\begin{itemize}
\item Initial conditions for wave equation:\\
\quad $u(0,x)=f(x)\qquad \partFrac{u}{t}=g(x)$
\item The functions $f$ and $g$ can also be represented as Fourier series:\\
$f(x)=\frac{a_0^f}{2}+\sum\limits_{k=1}^{\infty}{a^f_k\cos(kx)+b^f_k\sin(kx)}$\\[0.2cm]
$g(x)=\frac{a_0^g}{2}+\sum\limits_{k=1}^{\infty}{a^g_k\cos(kx)+b^g_k\sin(kx)}$
\item Together with the approach for $u(t,x)$, the equations (for $t=0$) are obtained:\\
$\frac{a_0(0)}{2}+\sum\limits_{k=1}^{\infty}{a_k(0)\cos(kx)+b_k(0)\sin(kx)}=\frac{a_0^f}{2}+\sum\limits_{k=1}^{\infty}{a_k^f\cos(kx)+b^f_k\sin(kx)}$\\[0.2cm]
$\frac{a'_0(0)}{2}+\sum\limits_{k=1}^{\infty}{a_k'(0)\cos(kx)+b_k'(0)\sin(kx)}=\frac{a_0^g}{2}+\sum\limits_{k=1}^{\infty}{a_k^g\cos(kx)+b^g_k\sin(kx)}$
\item Coefficient comparison yields:\\
$a_k(0)=a_k^f\qquad a_k'(0)=a_k^g\qquad b_k(0)=b_k^f\qquad b_k'(0)=b_k^g$
\item The complete solution is thus:\\
$u(t,x)=\frac{a_0^g(t)+a_0^f}2+\sum\limits_{k=1}^{\infty}{\left(a_k^f\cos(kt)+\frac 1k a_k^g\sin(kt)\right)\cos(kx)+\left(b_k^f\cos(kt)+\frac 1k b_k^g\sin(kt)\right)}\sin(kx)$
\end{itemize}

\subsubsection{Inhomogeneous Wave Equation}

The method can also be generalized to the inhomogeneous wave equation. The perturbation term is also expanded as a Fourier series.

$\partial_t^2u-\partial_x^2u=f \qquad \Rightarrow \qquad f(t,x)=\frac{a_0^f(t)}{2}+\sum\limits_{k=1}^{\infty}{a^f_k(t)\cos(kx)+b^f_k\sin(kx)}$

\subsubsection{Laplace Transform}

$\boxed{F(t)=\int\limits_{0}^{\infty}{f(t)\e^{-st}} dt}$ \qquad See also later in the summary!
(Chapter \ref{sec:Laplace Umwandulungen} on page \pageref{sec:Laplace Umwandulungen})\\

\textbf{Solution of an ODE:}\\

$\dot{x}(t)+p x(t)=f(t) \qquad f(t)=q$\\
$\dot{x}(t)+p x(t)=f(t)\FT s X(s)-x(0)+pX(s)=F(s) \qquad f(t)\FT F(s)=\frac{q}{s}$\\

$\Rightarrow X(s)=\frac{F(s)+x(0)}{s+p}=\frac{q+x(0)}{s(s+p)}\Big|_{x(0)=0}\IFT x(t)=\frac{q}{p}(1-\e^{-pt})$\\

\textbf{Solution of a PDE:}\\

$\partFrac{u}{t}+x\partFrac{u}{x}=x\qquad t\geq 0,\quad x\geq 0\qquad u(x,0)=0,\quad u(0,t)=0\qquad x,t>0$\\

Transformation: $\partFrac{u}{t}+x\partFrac{u}{x}=x\FT sU(s,x)-u(x,0)+x\partFrac{U(s,x)}{x}=\frac{x}{s}\qquad \Rightarrow \qquad U(s,x)=\frac{x}{s(s+1)}$\\
$U(s,x)\IFT x(1-\e^{-t})$

\subsubsection{Example: Heat Conduction}

At time $t=0$, the rod has temperatures of $-1$ at $x=-\frac{\pi}2$ and
$1$ at $x=\frac{\pi}2$ $\rightarrow$ stationary state.
At time $t=0$, the reservoirs are removed, and the rod is left to itself. In particular, no heat can be dissipated through the ends.
\[
\frac{\partial u}{\partial t}=\frac{\partial^2 u}{\partial x^2} \qquad \text{or in general:} \qquad \frac{\partial u}{\partial t}= a^2 \frac{\partial^2 u}{\partial x^2}
\]

``Triangle function''
\[
d(x)
=
\begin{cases}
\displaystyle-2-\frac{2x}{\pi}&\qquad \displaystyle-\frac{\pi}2\le x\\
\displaystyle\frac{2x}{\pi}&\qquad \displaystyle-\frac{\pi}2\le x\le \frac{\pi}2\\
\displaystyle2-\frac{2x}{\pi}&\qquad \displaystyle x\le\frac{\pi}2
\end{cases}
 \qquad = \qquad \sum_{n=0}^\infty \frac{8(-1)^n}{\pi^2(2n+1)^2}\sin ((2n+1)x)
\]

 $\hat u(t,k)$ Fourier-Sine-Coefficients / ${\cal L} u$ Laplace Transformation.

Initial conditions are odd $\rightarrow$ solution of the
differential equation for all times odd. \\
Boundary conditions:
$\partial_xu(t,-\frac{\pi}2)=\partial_xu(t,\frac{\pi}2)=0$
$\rightarrow$ Reflection at
$-\frac{\pi}2$ and $\frac{\pi}2$ $\rightarrow$ extend to a $2\pi$-periodic
function on $\mathbb R$
\[
\partial_t\hat u(t,k)=-k^2\hat u(t,k)
\]
Now this equation can be Laplace-transformed:
\begin{align*}
s{\cal L}\hat u(s,k) - \hat u(0,k)&=-k^2 {\cal L}\hat u(s,k)\\
(s+k^2){\cal L}\hat u(s,k)&=\hat u(0,k)\\
{\cal L}\hat u(s,k)&=\frac{\hat u(0,k)}{s+k^2}
\end{align*}
Inverse transformation yields:
\[
\hat u(t,k)=\hat u(0,k) e^{-k^2t}
\]
Now, only the Fourier coefficients need to be determined, which can be obtained from the triangle function:
\[
\hat u(0,2n+1)=
\frac{8(-1)^n}{\pi^2(2n+1)^2}
\]
and thus, the final solution is obtained by summing the Fourier series:
\[
u(t,x)=
\sum_{n=0}^\infty \frac{8(-1)^n}{\pi^2(2n+1)^2}e^{-(2n+1)^2t}\sin((2n+1)x)
\]

\subsection{PDGL 2.Ordnung}
Lineare partielle Differentialgleichungen zweiter Ordnung haben die Form:
$\boxed{\sum\limits_{i,j=1}^{n}{a_{ij}\partial_i\partial_j u}+\sum\limits_{i=1}^{n}{b_i\partial_i u}+cu=f}$

\subsubsection{Klassifikation}
Klassifikation nur für PDEs zweiter Ordnung!

\begin{minipage}{9cm}
  Eigenwertberechnung: (z.B. von $ \partial^2_xu+2\partial_x\partial_yu+\partial^2_yu=0 $) 
  \begin{enumerate}
    \item Symmetrische Matrix aufstellen und $\lambda$ in der Diagonalen abziehen. Z.B.: $A = \begin{pmatrix}
      \partial_x^2 & \partial_x \partial_y \\
      \partial_y \partial_x  & \partial_y^2
    \end{pmatrix}$\\
    Bei diagonalen Matrizen entsprechen die Eigenwerte den Diagonaleinträgen.
    \item Determinante gleich 0 setzen: $\det(\mathbf{A}-\lambda \mathbf{I}) = 0\quad\Rightarrow\quad \lambda_i$
    \item Gleichung lösen
  \end{enumerate}
\end{minipage}
\begin{minipage}{9cm}
  Alternativ (wenn z.B. sehr wüste PDE klassifiziert werden muss), können auch via Spur und Determinante die Vorzeichen der Eigenwerte herausgefunden werden:
  \begin{enumerate}
    \item Siehe links (Eigenwertberechnung): Matrix $A$ aufstellen
    \item Determinante berechnen und versuchen aus Tabelle zu lesen:
     $\det A = a_{11}a_{22} - a_{12}a_{21} = \lambda_1 \lambda_2$
    \item Spur berechnen und versuchen aus Tabelle zu lesen:
      $\tr(A) = a_{11} + a_{22} = \lambda_1 + \lambda_2$
  \end{enumerate}
  
\end{minipage}

\begin{center}
\begin{tabular}{|l||l|l|l|l|l|}
\hline
\multirow{2}{*}{Klasse}&\multicolumn{3}{|c|}{Anzahl Eigenwerte} & det(A)&\multirow{2}{*}{Beispiel}\\
& Positiv & Negativ & Verschwindend(=0) & für n=2 &\\
\hline
hyperbolisch& n-1 & 1 & 0 & det < 0 & Wellengleichung: $\frac{\partial^2 u}{\partial t^2} = \Delta u$ \\
\hline
parabolisch& n-1 & 0 & 1 & det = 0 & Wärmeleitung: $\frac{\partial u}{\partial t} = \Delta u$  \\
\hline
elliptisch&	n & 0 & 0 & det > 0 & Potential: $\Delta u = f$ \\
\hline
ultrahyperbolisch & >1 & >1 & 0 & - & -\\
\hline
\end{tabular}
\end{center}

\subsection{Elliptische PDGL}
$\Delta u=f\qquad \omega=\{(x,y)|y\geq 0\},\quad u(x,y)=ay$

\textbf{Satz:} Wenn $\Omega$ beschränkt und zusammenhängend, dann ist die Lösung u immer eindeutig.\\

\textbf{Beweis:} Annahme: $u=u_1-u_2$\\
Einsetzen: $\Delta u_1 - \Delta u_2=f-f=0$\\
$\left.(u_1-u_2)\right|_{\partial \omega}=g-g=0$\\
$\Delta u=0 \qquad \left.u\right|_{\partial\Omega}=0$\\
Falls $u=0$ eine Lösung, dann gibt es nur eine Lösung.

\subsubsection{Maximumprinzip} 

Wenn gilt $\Delta u=0$, so ist $u$ \emph{harmonisch}, und dann befinden 
sich die Extrema (Maxima und Minima der Funktion) auf dem Rand $\partial\Omega$.

\subsubsection{Beispiel (Übungslösungen)}
Eine elliptische PDGL wie $\Delta u = c$ hat mit der vorgegebenen
Dirichlet-Randwerten nur eine Lösung. Zur Erinnerung: Der Grund war das
Maximum-Prinzip. Gäbe es nämlich eine zweite Lösung $\bar v(r,\phi)$ mit
gleichen Randwerten, wäre $v - \bar v$ eine Lösung der Gleichung $\Delta (v -
\bar v) = 0$ also harmonische Funktion. Die Randwerte von $v - \bar v$ sind 0. Da eine
harmonische Funktion das Maximum auf dem Rand annimmt ist $v - \bar v = 0$ die
Lösung ist also eindeutig.

\newpage
\subsubsection{Greensche Funktion} 

Eine elliptische PDGL wird mittels Inversion von $\Delta$ gelöst. Dieser Umkehr geschieht mittels Greenscher Funktion, welche die Umkehrfunktion $\Delta$ ist\qquad $\Delta$: Laplace-Operator.

$u(x)=\int\limits_\Omega{\sigma(x,\xi)f(\xi)d\xi}+\int\limits_\Omega{h(x,\xi)f(\xi)d\xi}\qquad \sigma(x,\xi)=
\begin{cases}
	\frac 12|x-\xi| & n=1\\ 
	\frac 1{2\pi}\log|x-\xi| & n=2\\
	-\frac 1{4\pi}\frac{1}{|x-\xi|} & n=3\\
	\frac {1}{(2-n)\mu(S^{n-1})}|x-\xi|^{2-n} & n\geq 3\\
\end{cases}$\\


Greensche Funktion: $G(x,\xi)=\sigma(x,\xi)+h(x,\xi)$

Satz: Ist $\Omega$ ein Gebiet, auf dem das Dirichlet Problem eindeutig lösbar ist, dann gibt es eine Funktion $G(x,\xi)$, welche als Funktion von x die Gleichung

$\Delta G(x,\xi)=\delta(x-\xi)$

löst mit homogenen Randbedingungen.
Lösung: $u(x)=\int\limits_{\Omega}^{}{G(x,\xi)f(\xi)d\xi}+\int\limits_{\partial\Omega}g(\xi)\cdot\grad{\xi}G(x,\xi)d\eta\qquad \eta:\text{ Normale von }\partial G$

\subsubsection{Mittelwerteigenschaft harmonischer Funktionen}

$\Delta h=0$\qquad Mittelwerteigenschaft:\qquad $h(x)=
\begin{cases}
	\frac{h(x+\delta)+h(x-\delta)}{2}& n=1 \\ 
	\frac 1{2\pi r} \int\limits_{S_r^1}{h(x+\xi)d\xi} & n=2\\
	\frac 1{4\pi r^2} \int\limits_{S_r^2}{h(x+\xi)d\xi} & n=3\\
\end{cases}$
%\subsection{Parabolische PDGL}
\todo{Beschreibung}
\subsection{Hyperbolische PDGL}
\begin{minipage}{14cm}
	PDGL: $\partFrac{^2u}{t^2}-a^2\partFrac{^2 u}{x^2}=0\qquad \Omega=\left\{(x,t)|t>0\right\}\qquad u_0=u(x_0,0)$\\
	
	Trick: $(\partial_t +a\partial_x)(\partial_t-a\partial_x)u=(\partial_t^2-a^2\partial_x^2)u=0$\qquad(für konstante Geschwindigkeit $a$)\\
	
	Zwei mögliche Lösungen: $\underset{\text{\cfbox{red}{Nach rechts laufende Welle}}}{\underbrace{(\partial_t +a\partial_x)u=0}}\qquad \underset{\text{\cfbox{black}{Nach links laufende Welle}}}{\underbrace{(\partial_t-a\partial_x)u=0}}$\\
	
	Lösung mittels Charakteristiken: $\partFrac{}{s}
	\begin{Bmatrix}
		x(s)\\
		t(s)\\
		u(s)
	\end{Bmatrix}=
	\begin{Bmatrix}
		\pm a\\
		1\\
		0
	\end{Bmatrix}
	\begin{array}{ll}
		\Rightarrow&x=\pm as +x_0\\
		\Rightarrow&t= s +t_0=s\qquad (t_0=0)\\
		\Rightarrow&u=u_0\\
	\end{array}
	$\\
	
	$x=\pm at+x_0\quad\Rightarrow\quad x_0=x\mp at\quad\Rightarrow\quad u(x,t)=u_0(x\mp at)$\\
	
	Allgemeine Lösung aus Linearkombination: $\boxed{u(x,t)=u_+(x+at)+u_-(x-at)}$\\
	
	$\Rightarrow$ Es werden \textbf{zwei} Anfangsbedingungen benötigt um $u_+$ \textbf{und} $u_-$ zu bestimmen.\\
	
	z.B.: $u(x,0)=u_0(x)\qquad \partFrac{u}{t}(x,0)=g_0(x)$
	\end{minipage}
	\hfill
	\begin{minipage}{5cm}
	\includegraphics[width=5cm]{Content/Theory/linksRechts}
	\end{minipage}
\subsubsection{Streifen/Charakteristiken}
	PDGL: $a\partial_x^2u+2b\partial_x\partial_yu+c\partial_y^2u+d\partial_xu+e\partial_yu+fu=g$ Symbolmatrix: $ 
		\begin{bmatrix}
			a & b\\
			b & c
		\end{bmatrix}$
	
	Entlang der Kurve $t\mapsto(x(t),y(t))$ sind die Anfangswerte / partiellen Ableitungen
	$
	\left.
	\begin{aligned}
	u(x(t),y(t))&=u(t)\\
	\partial_xu(x(t),y(t))&=p(t)\\
	\partial_yu(x(t),y(t))&=q(t)
	\end{aligned}
	\qquad
	\right\}
	\label{charanfangs}
	$
	
	Charakteristiken erfüllen DGL:
    \[
        a\dot y(t)^2-2b\dot x(t)\dot y(t)+c\dot x(t)^2=0
    \]
	
	Charakteristischer Streifen erfüllt zusätzlich: $a\dot p(t)\dot y(t)-h\dot x(t)\dot y(t)+c\dot x(t)\dot q(t)=0$


