
\subsection{Elliptic PDEs}
$\Delta u=f\qquad \omega=\{(x,y)|y\geq 0\},\quad u(x,y)=ay$

\textbf{Theorem:} If $\Omega$ is bounded and connected, then the solution $u$ is always unique.\\

\textbf{Proof:} Assume: $u=u_1-u_2$\\
Substitute: $\Delta u_1 - \Delta u_2=f-f=0$\\
$\left.(u_1-u_2)\right|_{\partial \omega}=g-g=0$\\
$\Delta u=0 \qquad \left.u\right|_{\partial\Omega}=0$\\
If $u=0$ is a solution, then there is only one solution.

\subsubsection{Maximum Principle}

If $\Delta u=0$, then $u$ is \emph{harmonic}, and the extrema (maxima and minima of the function) are located on the boundary $\partial\Omega$.

\subsubsection{Example (Exercise Solutions)}
An elliptic PDE like $\Delta u = c$ has only one solution with the given Dirichlet boundary values. Remember: The reason was the maximum principle. If there were a second solution $\bar v(r,\phi)$ with the same boundary values, $v - \bar v$ would be a solution to the equation $\Delta (v - \bar v) = 0$, a harmonic function. The boundary values of $v - \bar v$ are 0. Since a harmonic function attains the maximum on the boundary, $v - \bar v = 0$ is the solution, so it is unique.

\newpage
\subsubsection{Green's Function}

An elliptic PDE is solved by inverting $\Delta$. This inversion is done using Green's function, which is the inverse function of $\Delta$ \qquad $\Delta$: Laplace operator.

$u(x)=\int\limits_\Omega{\sigma(x,\xi)f(\xi)d\xi}+\int\limits_\Omega{h(x,\xi)f(\xi)d\xi}\qquad \sigma(x,\xi)=
\begin{cases}
	\frac 12|x-\xi| & n=1\\
	\frac 1{2\pi}\log|x-\xi| & n=2\\
	-\frac 1{4\pi}\frac{1}{|x-\xi|} & n=3\\
	\frac {1}{(2-n)\mu(S^{n-1})}|x-\xi|^{2-n} & n\geq 3\\
\end{cases}$\\


Green's function: $G(x,\xi)=\sigma(x,\xi)+h(x,\xi)$

Theorem: If $\Omega$ is a domain where the Dirichlet problem is uniquely solvable, then there exists a function $G(x,\xi)$, which, as a function of $x$, solves the equation
$\Delta G(x,\xi)=\delta(x-\xi)$
with homogeneous boundary conditions.
Solution: $u(x)=\int\limits_{\Omega}^{}{G(x,\xi)f(\xi)d\xi}+\int\limits_{\partial\Omega}g(\xi)\cdot\grad{\xi}G(x,\xi)d\eta\qquad \eta:$ normal vector of $\partial G$

\subsubsection{Mean Value Property of Harmonic Functions}

$\Delta h=0$\qquad Mean Value Property:\qquad $h(x)=
\begin{cases}
	\frac{h(x+\delta)+h(x-\delta)}{2}& n=1 \\
	\frac 1{2\pi r} \int\limits_{S_r^1}{h(x+\xi)d\xi} & n=2\\
	\frac 1{4\pi r^2} \int\limits_{S_r^2}{h(x+\xi)d\xi} & n=3\\
\end{cases}$
