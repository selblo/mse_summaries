
\subsection{Transformations}
\begin{itemize}
\item The transition from functions to Fourier series transforms a partial differential equation into a family of ordinary differential equations for the individual Fourier coefficients.
\item Integral transformations can transform a partial differential equation into a family of partial differential equations with fewer variables or even ordinary differential equations.
\item Integral transformations and their inverses can provide formulas for solutions to certain partial differential equations, answering the question of well-posedness for given boundary conditions.
\end{itemize}

\begin{tabular}{ll}
  Domain & Transformation \\
  \hline
  $[0, \infty)$ & Laplace transformation \\
  $\mathbb{R}$ & Fourier transformation \\
  $[- \pi, \pi]$ & Fourier series \\
\end{tabular}

\subsubsection{Fourier Series}
$\boxed{u(t,x)=\frac{a_0(t)}{2}+\sum\limits_{k=1}^{\infty}{a_k(t)\cos(kx)+b_k(t)\sin(kx)}}$\\[0.4cm]

\subsubsection{Example: Vibrating String}

$\boxed{\partial_t^2u=\partial_x^2u}$

\begin{enumerate}
\item Insert Fourier analysis approach into PDE:\\
$$\partial_t^2(t,x)=\frac{a_0''(t)}{2}+\sum\limits_{k=1}^{\infty}{a_k''(t)\cos(kx)+b_k''(t)\sin(kx)}
\qquad \qquad
\partial_x^2(t,x)=-\sum\limits_{k=1}^{\infty}{a_k(t)k^2\cos(kx)+b_k(t)k^2\sin(kx)}$$
$$\partial_t^2(t,x)=\partial_x^2(t,x)
\qquad \Longleftrightarrow \qquad \frac{a_0''(t)}{2}+\sum\limits_{k=1}^{\infty}{a_k''(t)\cos(kx)+b_k''(t)\sin(kx)}=-\sum\limits_{k=1}^{\infty}{a_k(t)k^2\cos(kx)+b_k(t)k^2\sin(kx)}$$
$\boxed{\Rightarrow\quad \frac{a_0''(t)}{2}+\sum\limits_{k=1}^{\infty}{\big(a_k''(t)+a_k(t)k^2\big)\cos(kx)+\big(b_k''(t)+b_k(t)k^2\big)\sin(kx)}=0}$
\item This equation is only solvable if all coefficients vanish (Fourier theory):\\[0.2cm]
$a_0''(t)=0 \qquad a_k''(t)=-k^2a_k(t)\qquad b_k''(t)=-k^2b_k(t)$
\item The Fourier transformation has transformed the PDE into a system of ODEs, the solutions of which are well-known:\\[0.2cm]
$a_0(t)=m_0(t)+c_0\qquad a_k(t)=A_k^a\cos(kt)+B_k^a\sin(kt)\qquad b_k(t)=A_k^b\cos(kt)+B_k^b\sin(kt)$
\end{enumerate}

\textbf{Initial Conditions:}
The differential equations for the coefficients $a_k(t)$ and $b_k(t)$ can only be completely solved when initial or boundary conditions are given.
\begin{itemize}
\item Initial conditions for wave equation:\\
\quad $u(0,x)=f(x)\qquad \partFrac{u}{t}=g(x)$
\item The functions $f$ and $g$ can also be represented as Fourier series:\\
$f(x)=\frac{a_0^f}{2}+\sum\limits_{k=1}^{\infty}{a^f_k\cos(kx)+b^f_k\sin(kx)}$\\[0.2cm]
$g(x)=\frac{a_0^g}{2}+\sum\limits_{k=1}^{\infty}{a^g_k\cos(kx)+b^g_k\sin(kx)}$
\item Together with the approach for $u(t,x)$, the equations (for $t=0$) are obtained:\\
$\frac{a_0(0)}{2}+\sum\limits_{k=1}^{\infty}{a_k(0)\cos(kx)+b_k(0)\sin(kx)}=\frac{a_0^f}{2}+\sum\limits_{k=1}^{\infty}{a_k^f\cos(kx)+b^f_k\sin(kx)}$\\[0.2cm]
$\frac{a'_0(0)}{2}+\sum\limits_{k=1}^{\infty}{a_k'(0)\cos(kx)+b_k'(0)\sin(kx)}=\frac{a_0^g}{2}+\sum\limits_{k=1}^{\infty}{a_k^g\cos(kx)+b^g_k\sin(kx)}$
\item Coefficient comparison yields:\\
$a_k(0)=a_k^f\qquad a_k'(0)=a_k^g\qquad b_k(0)=b_k^f\qquad b_k'(0)=b_k^g$
\item The complete solution is thus:\\
$u(t,x)=\frac{a_0^g(t)+a_0^f}2+\sum\limits_{k=1}^{\infty}{\left(a_k^f\cos(kt)+\frac 1k a_k^g\sin(kt)\right)\cos(kx)+\left(b_k^f\cos(kt)+\frac 1k b_k^g\sin(kt)\right)}\sin(kx)$
\end{itemize}

\subsubsection{Inhomogeneous Wave Equation}

The method can also be generalized to the inhomogeneous wave equation. The perturbation term is also expanded as a Fourier series.

$\partial_t^2u-\partial_x^2u=f \qquad \Rightarrow \qquad f(t,x)=\frac{a_0^f(t)}{2}+\sum\limits_{k=1}^{\infty}{a^f_k(t)\cos(kx)+b^f_k\sin(kx)}$

\subsubsection{Laplace Transform}

$\boxed{F(t)=\int\limits_{0}^{\infty}{f(t)\e^{-st}} dt}$ \qquad See also later in the summary!
(Chapter \ref{sec:Laplace Umwandulungen} on page \pageref{sec:Laplace Umwandulungen})\\

\textbf{Solution of an ODE:}\\

$\dot{x}(t)+p x(t)=f(t) \qquad f(t)=q$\\
$\dot{x}(t)+p x(t)=f(t)\FT s X(s)-x(0)+pX(s)=F(s) \qquad f(t)\FT F(s)=\frac{q}{s}$\\

$\Rightarrow X(s)=\frac{F(s)+x(0)}{s+p}=\frac{q+x(0)}{s(s+p)}\Big|_{x(0)=0}\IFT x(t)=\frac{q}{p}(1-\e^{-pt})$\\

\textbf{Solution of a PDE:}\\

$\partFrac{u}{t}+x\partFrac{u}{x}=x\qquad t\geq 0,\quad x\geq 0\qquad u(x,0)=0,\quad u(0,t)=0\qquad x,t>0$\\

Transformation: $\partFrac{u}{t}+x\partFrac{u}{x}=x\FT sU(s,x)-u(x,0)+x\partFrac{U(s,x)}{x}=\frac{x}{s}\qquad \Rightarrow \qquad U(s,x)=\frac{x}{s(s+1)}$\\
$U(s,x)\IFT x(1-\e^{-t})$

\subsubsection{Example: Heat Conduction}

At time $t=0$, the rod has temperatures of $-1$ at $x=-\frac{\pi}2$ and
$1$ at $x=\frac{\pi}2$ $\rightarrow$ stationary state.
At time $t=0$, the reservoirs are removed, and the rod is left to itself. In particular, no heat can be dissipated through the ends.
\[
\frac{\partial u}{\partial t}=\frac{\partial^2 u}{\partial x^2} \qquad \text{or in general:} \qquad \frac{\partial u}{\partial t}= a^2 \frac{\partial^2 u}{\partial x^2}
\]

``Triangle function''
\[
d(x)
=
\begin{cases}
\displaystyle-2-\frac{2x}{\pi}&\qquad \displaystyle-\frac{\pi}2\le x\\
\displaystyle\frac{2x}{\pi}&\qquad \displaystyle-\frac{\pi}2\le x\le \frac{\pi}2\\
\displaystyle2-\frac{2x}{\pi}&\qquad \displaystyle x\le\frac{\pi}2
\end{cases}
 \qquad = \qquad \sum_{n=0}^\infty \frac{8(-1)^n}{\pi^2(2n+1)^2}\sin ((2n+1)x)
\]

 $\hat u(t,k)$ Fourier-Sine-Coefficients / ${\cal L} u$ Laplace Transformation.

Initial conditions are odd $\rightarrow$ solution of the
differential equation for all times odd. \\
Boundary conditions:
$\partial_xu(t,-\frac{\pi}2)=\partial_xu(t,\frac{\pi}2)=0$
$\rightarrow$ Reflection at
$-\frac{\pi}2$ and $\frac{\pi}2$ $\rightarrow$ extend to a $2\pi$-periodic
function on $\mathbb R$
\[
\partial_t\hat u(t,k)=-k^2\hat u(t,k)
\]
Now this equation can be Laplace-transformed:
\begin{align*}
s{\cal L}\hat u(s,k) - \hat u(0,k)&=-k^2 {\cal L}\hat u(s,k)\\
(s+k^2){\cal L}\hat u(s,k)&=\hat u(0,k)\\
{\cal L}\hat u(s,k)&=\frac{\hat u(0,k)}{s+k^2}
\end{align*}
Inverse transformation yields:
\[
\hat u(t,k)=\hat u(0,k) e^{-k^2t}
\]
Now, only the Fourier coefficients need to be determined, which can be obtained from the triangle function:
\[
\hat u(0,2n+1)=
\frac{8(-1)^n}{\pi^2(2n+1)^2}
\]
and thus, the final solution is obtained by summing the Fourier series:
\[
u(t,x)=
\sum_{n=0}^\infty \frac{8(-1)^n}{\pi^2(2n+1)^2}e^{-(2n+1)^2t}\sin((2n+1)x)
\]
